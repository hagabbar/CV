\section{\sc Research Experience}

{\bf{Institute for Gravitational Research, University of Glasgow}}\\
\begin{tabular}{@{}p{4in}p{2in}}
Ph.D. Student, Researcher & 07/2017 - present\\
\end{tabular}
\begin{itemize}
\setlength\itemsep{0em}
\item Mr. Gabbard is working on the direct application of cutting edge machine learning techniques for the purposes of identifying and characterising gravitational wave signals in data generated by the LIGO-Virgo Collaboration. Specifically he is developing convolutional neural networks to classify the presence of signals from the mergers of binary black holes and/or neutron stars. This task is complicated by the presence of unknown non-astrophysical signals (or detector "glitches") which must also be identified and characterised. The glitch analysis will be enhanced by the inclusion of additional environmental/diagnostic datasets generated at the detector sites. A key component of the analysis is to also be able to rapidly identify the astrophysical source parameters and their corresponding uncertainties in any merger detections. This low-latency information is vital for the electromagnetic follow-up observations of gravitational wave events.
\end{itemize}

{\bf{Max Planck Institute for Gravitational Physics (Hannover, Germany)}}\\
\begin{tabular}{@{}p{4in}p{2in}}
Fulbright Scholar & 09/2016 - 07/2017\\
\end{tabular}
\begin{itemize}
\setlength\itemsep{0em}
\item Mr. Gabbard's contribution was to account for non-astrophysical noise. Mr. Gabbard's work utilized novel machine learning methods (i.e. deep sequential neural networks) in order to better characterize the significance of non-astrophysical noise, and specifically its effect on the search for merging binary neutron stars and black-holes.

Additionally, using deep learning classification methods, Mr. Gabbard spearheaded the development of a new complimentary detection statistic for the CBC pycbc search pipeline.
\end{itemize}

{\bf{University of Mississippi, Laser Interferometer Gravitational-Wave Observatory (LIGO) group
}}\\
\begin{tabular}{@{}p{4in}p{2in}}
Research Assistant & 01/2013 - 05/2016\\
\end{tabular}
\begin{itemize}
\setlength\itemsep{0em}
\item Transient investigations of observation run LIGO data and Advanced LIGO subsystems under the supervision of Dr. M.Cavagli\`{a}. Author of the ``Terramon'' monitor used in the LIGO control rooms to help predict the effects of seismic events at the observatories.
\item During the Summer of 2016, Mr. Gabbard collaborated with Kai Staats and Marco Cavaglia on the development of a low-latency glitch classification pipeline. This pipeline applies the use of a novel machine learning method called genetic programming (karoo\_gp, Kai Staats). During the training process, multivariate expressions are produced in a stochastic form and compete with one another. The most precise expression is chosen and then used to classify new glitches.
\end{itemize}

{\bf{University of Texas Rio Grande Valley}}\\
\begin{tabular}{@{}p{4in}p{2in}}
REU Research Assistant & 05/2015 - 08/2015\\
\end{tabular}
\begin{itemize}
\setlength\itemsep{0em}
\item Development of a low-latency glitch classification algorithm using unsupervised neural networks based in waveform morphology under the supervision of Dr. Soma Mukherjee.
\end{itemize}

{\bf{University of Florida and Laboratoire de l'Accelerateur Lineaire}}\\
\begin{tabular}{@{}p{4in}p{2in}}
Gravitational Physics IREU Research Assistant & 05/2014 - 08/2014\\
\end{tabular}
\begin{itemize}
\setlength\itemsep{0em}
\item Characterization of the Omicron trigger generator and transient analysis of aLIGO data under the supervision of Dr. Florent Robinet.
\\
\end{itemize}

\endinput